\chapter{The display container}

In order to realize an efficient OP list backend, it is needed to
distinguish the abstract notion of sprites from the notion of OP
sprites.

The \emph{display} container offers a high-level view of what will be
displayed by the Jaguar.

\section{The display container}
Logically, a display is divided in several \emph{layers}\footnote{in
  practice, there are 16 layers but this number can be modified if
  needed by recompiling the library with different parameters}. Each
layer can contain zero, one or more \emph{sprites}. The content of a
layer can ---of course--- evolves dynamically.

\smallskip

For instance \Fref{final} shows a display composed of three layers:
\begin{enumerate}
\item A background layer \Fref{layer1} that contains one sprite (the
  image with the sun and the grass)
\item A foreground layer \Fref{layer2} that contains a transparent
  sprite (the house)
\item A layer \Fref{layer3} that contains two sprites (the two
  characters)
\end{enumerate}

The three layers are superimposed as shown in \Fref{over}, yielding
the image of \Fref{final}.

\begin{figure}[htbp]
  \centering
  \input layer1  
  \caption{The background}
  \label{fig:layer1}
\end{figure}

\begin{figure}[htbp]
  \centering
  \input layer2
  \caption{The foreground}
  \label{fig:layer2}
\end{figure}

\begin{figure}[htbp]
  \centering
  \input layer3
  \caption{The characters}
  \label{fig:layer3}
\end{figure}

\begin{figure}[htbp]
  \centering
  \input over
  \caption{The three layers superimposed}
  \label{fig:over}
\end{figure}

\begin{figure}[htbp]
  \centering
  \input final
  \caption{The final result}
  \label{fig:final}
\end{figure}

\section{The sprite data structure}
A sprite can have several properties.


