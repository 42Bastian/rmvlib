\chapter{About scaled sprites}

\section{Scope}
A scaled sprite is characterised by:
\begin{itemize}
\item its width \wid{}
\item its height \hei{}
\item its horizontal scaling factor \hsc{}. This is a 3.5 fixed point
  integer.
\item its vertical scaling factor \vsc{}. This is a 3.5 fixed point
  integer.
\item its so called remainder \rem{}. This is a 3.5 fixed point
  integer.
\end{itemize}

According to \verb/Jag_v8.pdf/, we have:\\
{\small After each display line is drawn the \rem{} is
  decremented by one. If it becomes negative then \vsc{} is added to
  the \rem{} until it becomes positive. \hei{} is decremented every
  time \vsc{} is added to \rem{}.}\\

The problem is to compute the value of the remainder after $n$ lines
have been processed by the Object Processor and the corresponding line
in the source image of the sprite.

\section{Formalization}
Let $r \geq 0$ and $v > 0$.

We define the function $\K : \R \to \N$ by:
\begin{center}
  if $x \in \R$, then $\K(x) \Def \min \{ i \in \N \mid x - 1 + i v \geq 0 \}$
\end{center}
It is obvious that $\K$ is well-defined since $v > 0$. Moreover, if $x \geq 1$ then $\K(x) = 0$.

We define \suite{r} as follows:
\begin{align*}
  r_0 & \Def r \\
  r_{n+1} & \Def r_n - 1 + \K(r_n) v 
\end{align*}

We also define \suite{y} as follows:
\begin{align*}
  y_0 & \Def 0 \\
  y_{n+1} & \Def y_n + \K(r_n)
\end{align*}

The problem is: given $n \in \N$, is there a smarter way to compute
the values of $r_n$ and $y_n$ than using the recursive definitions?

\section{Solution}
\begin{lemma}
  \label{lem:lem1}
  \begin{enumerate}
  \item $\forall n: r_n \geq 0$
  \item $\forall n: y_n \geq 0$
  \end{enumerate}
\end{lemma}

\begin{lemma}
  \label{lem:lem2}
  $$\forall n: r_{n+1} = r_0 - (n+1) + y_{n+1} v$$
  \begin{proof}
    By induction on $n$.

    Indeed, we have
    \begin{align*}
      r_{n+1} & = r_n - 1 + (y_{n+1} - y_n) v \\
      & = r_{n-1} - 1 + (y_n - y_{n-1}) v - 1 + (y_{n+1} - y_n) v \\
      & = r_{n-1} - 2 + (y_{n+1} - y_{n-1}) v \\
      & = \ldots \\
      & = r_0 - (n+1) + (y_{n+1} - y_0) v \\
      & = r_0 - (n+1) + y_{n+1} v \\
    \end{align*}
  \end{proof}
\end{lemma}

\begin{theorem}
  \label{thm:thm1}
  $$\forall n: r_0 - (n+1) = y_{n+1} \times (- v) + r_{n+1}$$
\end{theorem}

\begin{lemma}
  \label{lem:lem3}
  If $r_n < 1$ then $r_{n+1} < v$
  \begin{proof}
    Assume that $r_n < 1$. Then $r_n - 1 < 0$ thus $\K(r_n) > 0$.
    
    We show that $r_{n+1} < v$.
    
    By definition, $r_{n+1} = r_n - 1 + \K(r_n) v$. By contradiction,
    assume that $r_{n+1} \geq v$. Then $r_{n+1} - v \geq 0$, \ie $r_n - 1
    + (\K(r_n) - 1) v \geq 0$. This is absurd since $\K(r_n)$ is
    the minimal integer $k$ such that $r_n - 1 + k v \geq 0$.
    
    Thus $0 \leq r_{n+1} < v$.
  \end{proof}
\end{lemma}    

\begin{lemma}
  \label{lem:lem4}
  If $r_n < v$ then $r_{n+1} < v$.
  \begin{proof}
    There are two cases.

    Either $r_n < 1$ and then $r_{n+1} < v$ by \Lref{lem3}.

    Or $1 \leq r_n < v$. Then $r_{n+1} = r_n - 1 < v$.
  \end{proof}
\end{lemma}

\begin{lemma}
  \label{lem:lem5}
  Let $n \in \N$.

  \begin{enumerate}
  \item if $n \leq \floor{r_0}$, then $r_n = r_0 - n$.
  \item if $n > \floor{r_0}$, then $0 \leq r_n < v$.
  \end{enumerate}
  \begin{proof}
    This follows from \Lref{lem4} and from the observation that if
    $r_n \geq 1$ then $r_{n+1} = r_n - 1$.
  \end{proof}
\end{lemma}

Let $n > \floor{r_0}$. Then by \Tref{thm1}, we have 
$$r_0 - n = y_n \times (-v) + r_n \text{ with } 0 \leq r_n < v$$

Thus, we have $\frac{r_0 - n}{v} - (- y_n) = \frac{r_n}{v}$, hence $0
\leq \frac{r_0 - n}{v} - (- y_n) < 1$. 

In other words, $- y_n = \floor{\frac{r_0 - n}{v}}$ when $n > \floor{r_0}$.

Thus, when $n > \floor{r_0}$, $y_n = - \floor{\frac{r_0 - n}{v}}$ and
$r_n = r_0 - n + y_n v$.

\bigskip

\fbox{In the sequel, we assume that $r_0 \in \Q$ and $v \in \Q$.}

\medskip

Then there exists $d\footnote{in our concrete case, $d = 2^5$} \in \N$
such that $d r_0 \in \N$ and $d v \in \N$.

We recall that
\begin{theorem}[Euclide's division]
  Let $a, b \in \Z$ with $b \neq 0$. There exists a unique $(q,r) \in
  \Z \times \N$ such that $a = b q + r$ with $0 \leq r < \abs{b}$.

  We say that $q$ is the \emph{quotient} of the division and $r$ is the
  \emph{remainder} of the division.
\end{theorem}

\bigskip

We thus have for $n \geq \floor{r_0}$ that

$$d r_0 - d n = y_n \times (- d v) + d r_n$$

with $(d r_0 - d n) \in \Z$, $- d v \in \Z$, $- d v \neq 0$ and $0 \leq
d r_n < \abs{- d v} = d v$.

\smallskip 

Hence, when $n > \floor{r_0}$,
\begin{itemize}
\item $y_n$ is the quotient of the division of $d r_0 - d n$ by $- d
  v$, and
\item $d r_n$ is the remainder of the division of $d r_0 - d n$ by $- d
  v$.
\end{itemize}

Note that when $n > \floor{r_0}$, we have $r_0 - n < 0$. 

Note also that we have $- v < 0$ since $v > 0$.

\smallskip

Recall finally that when $n \leq \floor{r_0}$,
\begin{itemize}
\item $y_n = 0$
\item $r_n = r_0 - n$
\end{itemize}

\section{In reality}
Unfortunately, the Atari Jaguar documentation is not accurate. So
after looking at the NET files (in particular WBK.NET), we realize
that instead we have:

\begin{align*}
  r_0 & \Def r \\
  r_{n+1} & \Def r_n - 1 + \K(r_n - 1) v 
\end{align*}

and
\begin{align*}
  y_0 & \Def 0 \\
  y_{n+1} & \Def y_n + \K(r_n - 1)
\end{align*}


Fortunately, we can reuse the previous study. Indeed, let \suite{r'}
be defined for any $n$ by $r'_n \Def r_n - 1$ and \suite{y'} be
defined by:
\begin{align*}
  y'_0 & \Def 0 \\
  y'_{n+1} & \Def y'_n + \K(r'_n)
\end{align*}

We clearly have that $y'_n = y_n$ for any $n$.

Moreover, we have $r'_0 = r_0 - 1$ and for any $n$
\begin{align*}
  r'_{n+1} & = r_{n+1} - 1 \\
  & = r_n - 1 + \K(r_n - 1) v - 1 \\
  & = r'_n - 1 + \K(r'_n)
\end{align*}

So we can apply the previous study to compute \suite{r'} and
\suite{y'} and deduce \suite{r} and \suite{y}.


