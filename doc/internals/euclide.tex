\chapter{About Euclide's division}

In $\Z$, we have
\begin{theorem}[Euclide's division]
  Let $a, b \in \Z$ with $b \neq 0$. There exists a unique $(q,r) \in
  \Z \times \N$ such that $a = b q + r$ with $0 \leq r < \abs{b}$.
\end{theorem}

In $\N$, we have
\begin{theorem}[Euclide's division]
  Let $a, b \in \N$ with $b \neq 0$. There exists a unique $(q,r) \in
  \N \times \N$ such that $a = b q + r$ with $0 \leq r < b$.
\end{theorem}

In both cases, we write $q = a / b$ and $r = a \% b$. Our problem now
is to express Euclide's division in $\Z$ as a function of Euclide's
division in $\N$.

So, let $a, b \in \Z$ with $b \neq 0$. 

There are four cases that are studied afterwards.

\section{$a \geq 0$ and $b > 0$}
Then it is Euclide's division in $\N$.

\section{$a \leq 0$ and $b > 0$}
In this case, $-a \in \N$.

Let $q = (-a) / b$ and $r = (-a) \% b$.

We have $-a = b q + r$ with $0 \leq r < b$.

Thus $a = b \times (-q) - r$.

\begin{itemize}
\item If $r = 0$ then $a = b \times (-q)$.

  So
  \begin{align*}
    a / b & = - ((-a) / b) \\
    a \% b & = (-a) \% b
  \end{align*}

\item If $r > 0$ then $0 < b - r < b = \abs{b}$.

  We have 
  \begin{align*}
    a & = a - b + b \\
    & = b \times (-q) -b + b - r \\
    & = b \times (-q-1) + b - r
  \end{align*}

  So
  \begin{align*}
    a / b & = - ((-a) / b) - 1 \\
    a \% b & = b - ((-a) \% b)
  \end{align*}
\end{itemize}

\section{$a \geq 0$ and $b < 0$}

  Then $-b > 0$ so $-b \in \N$.

  Let $q = a / (-b)$ and $r = a \% (-b)$.

  We have $a = (-b) \times q + r$ with $0 \leq r < -b$.

  Thus $a = b \times (-q) + r$ with $0 \leq r < \abs{b}$.

  So
  \begin{align*}
    a / b & = - (a / (-b)) \\
    a \% b & = a \% (-b)
  \end{align*}

\section{$a \leq 0$ and $b < 0$}

  Then $-a \in \N$ and $-b \in \N$.

  Let $q = (-a) / (-b)$ and $r = (-a) \% (-b)$.

  We have $-a = (-b) \times q + r$ with $0 \leq r < -b$.

  Thus $a = b q - r$.

  \begin{itemize}
  \item If $r = 0$ then $a = b q$.

    So 
    \begin{align*}
      a / b & = (-a) / (-b) \\
      a \% b & = (-a) \% (-b)
    \end{align*}

  \item If $r > 0$ then $0 < -r - b < -b = \abs{b}$.

    Thus
    \begin{align*}
      a & = a + b - b \\
      & = b q + b - b - r \\
      & = b (q+1) + (-b - r)
    \end{align*}

    So
    \begin{align*}
      a / b & = ((-a)/(-b)) + 1 \\
      a \% b & = - b - ((-a)\%(-b))
    \end{align*}
  \end{itemize}

\section{Summary}
To summarise, we have
\begin{center}
  \begin{tabular}{|c|c|c|}
    \hline
    \backslashbox{$a$}{$b$} & $b > 0$ & $b < 0$ \\
    \hline
    $a \geq 0$ &
    \begin{math}
      \begin{array}{l}
        q \leftarrow a / b \\
        r \leftarrow a \% b \\
      \end{array}
    \end{math}
    &        
    \begin{math}
      \begin{array}{l}
        q \leftarrow a / (-b) \\
        r \leftarrow a \% (-b) \\
        q \leftarrow -q \\
      \end{array}
    \end{math}
    \\ 
    \hline
    $a \leq 0$ &
    \begin{math}
      \begin{array}{l}
        q \leftarrow (-a) / b \\
        r \leftarrow (-a) \% b \\
        q \leftarrow -q \\
        r \leftarrow -r \\
        \textbf{if $r < 0$ then} \\
        \quad q \leftarrow q - 1 \\
        \quad r \leftarrow r + b \\
      \end{array}
    \end{math}
    &
    \begin{math}
      \begin{array}{l}
        q \leftarrow (-a) / (-b) \\
        r \leftarrow (-a) \% (-b) \\
        r \leftarrow -r \\
        \textbf{if $r < 0$ then} \\
        \quad q \leftarrow q + 1 \\
        \quad r \leftarrow r - b \\
      \end{array}
    \end{math}
    \\
    \hline
  \end{tabular}
\end{center}
