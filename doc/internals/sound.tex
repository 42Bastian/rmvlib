\chapter{The sound engine}

\newcommand{\fname}[1]{\textsc{#1}}

\section{Paula emulator, version 1.0}
The Paula emulator is able to resample and mix 8 independent stereo
voices.

A global frequency $f_r$ is assumed. It is the frequency at which
sound data are sent to the Jaguar DACs.

\subsection{Voice description}
Each voice is described thanks to several fields. The table below
lists all the (public) fields that describe a voice and summarise when
the Paula emulator access these fields.

\begin{center}
  \begin{tabular}{|l|l|l|}
    \hline
    field name & brief description & read access \\
    \hline
    \fname{START} & start address of sound data & whenever the voice loops\\
    \fname{LENGTH} & length of sound data in bytes & whenever the
    voice loops\\
    \fname{CONTROL} & replay characteristics (see below) & as long as
    the voice is enabled\\
    \hline
  \end{tabular}
\end{center}

If \fname{START} becomes NULL, then the voice will be disabled when
the voice would have looped.

The \fname{CONTROL} field defines the replay characteristics of the
voice:
\begin{itemize}
\item format of sound data: 8 or 16 bits (signed, big endian).
\item left/right balance (panning) defined by an integer $0 \leq b
  \leq 16$
\item its volume defined by an integer $0 \leq v \leq 64$ 
  (note: the scale is linear, which is incorrect)
\item its replay frequency, which is defined by an increment (12 bits
  fixed point) $0 \leq i \leq 16$.

  If $f_s$ is the frequency of the sound and $f_r$ is the global
  frequency of the sound driver, the increment is given by
  the formula
  \begin{center}
    \fbox{$i = \frac{f_s}{f_r}$}
  \end{center}
\end{itemize}
Its value may change while a voice is enabled.

\subsection{Programming the Paula emulator}

In order to control when the Paula emulator needs to take into account
new commands written in the voice descriptors a global \fname{CONTROL}
field is used to communicate is the Paula emulator. Any process can
read/write this field (including the Paula emulator). 

Before writing this field, any external process should wait that its
value becomes NULL.

When the Paula emulator reads a non NULL value here, it will take into
account the command and then write back a NULL value.

The global \fname{CONTROL} is defined as follows:
\begin{itemize}
\item each voice has a corresponding bit. The meaning of this bit is
  the following:
  \begin{itemize}
  \item if the bit is clear, the command does not affect the
    corresponding voice (the voice is \emph{unmarked}).
  \item if the bit is set, the command does affect the corresponding
    voice (the voice is \emph{marked}).
  \end{itemize}
\item a global bit indicates the meaning of the command:
  \begin{itemize}
  \item if this bit is clear, all the marked voices are disabled.
  \item if this bit is set, all the marked voices are enabled. The
    fields \fname{START} and \fname{LENGTH} of the marked voices are
    read at that point to initialize the voice.
  \end{itemize}
\end{itemize}

\subsection{How to?}

\paragraph{Disabling a voice}
\begin{enumerate}
\item Wait that the global \fname{CONTROL} becomes NULL
\item Mark the voice to disable and write new value of \fname{CONTROL}
\end{enumerate}

\paragraph{Enabling a voice (one shot replay)}
\begin{enumerate}
\item Fill the voice characteristics (\fname{START}, \fname{LENGTH}, \fname{CONTROL})
\item Wait that the global \fname{CONTROL} becomes NULL
\item Mark the voice to enable and write new value of \fname{CONTROL}
\item Disable replay loop by writing NULL to field \fname{START} of
  the voice.
\end{enumerate}

Note that if the voice is already in use, the above process might be
incorrect. It is safer to first disable the voice, then enable it
again.

\paragraph{Enabling a voice (replay with loop)}
\begin{enumerate}
\item Fill the voice characteristics (\fname{START}, \fname{LENGTH}, \fname{CONTROL})
\item Wait that the global \fname{CONTROL} becomes NULL
\item Mark the voice to enable and write new value of \fname{CONTROL}
\item Fill the start of the replay loop and the length of the loop
  (\fname{START}, \fname{LENGTH})
\end{enumerate}
